\chapter{Methods and Resources}
The imported dataset for all computations is provided by the \href{https://keras.io/api/datasets/mnist/}{keras.datasets} package.
All python scripts include the \href{https://docs.python.org/3/library/datetime.html}{datetime} and \href{https://docs.python.org/3/library/time.html}{time} packages for timing purposes, while \href{https://matplotlib.org/}{matplotlib} offers some visualization functionalities for datasets.
With the implementation of 4 python programs, each script uses its own respective set of packages.

\begin{enumerate}
    \item \href{https://scikit-learn.org/stable/modules/generated/sklearn.svm.SVC.html}{SKLearn support vector machine} (SVM)
    \item \href{https://scikit-learn.org/stable/modules/generated/sklearn.svm.SVC.html}{SKLearn support vector machine} with \href{https://scikit-learn.org/stable/modules/generated/sklearn.decomposition.PCA.html}{principal component analysis} (PCA)
    \item \href{https://scikit-learn.org/stable/modules/generated/sklearn.neural_network.MLPClassifier.html}{SKLearn multi-layer perceptron classification}, with and without PCA
    \item \href{https://keras.io/guides/sequential_model/}{Keras Sequential model}
\end{enumerate}

The programs implementing the SKLearn-Kit library further allow for hyperparameter search using GridSearchCV, which allows automated customization and inspection of each algorithm over various pre-defined parameters in search of an optimal score for specific criteria, here ''accuracy''. The script using the Keras library has additional information on which images were especially difficult to classify.

\subsubsection{Approach}
After fetching the data from the Keras database, it is necessary to reshape and normalize the data for machine learning purposes.
The reshaping of the data is the transformation of the original three-dimensional dataset into a two-dimensional training set.
We reshape the data so that we have access to every pixel of the image.
The reason to access every pixel is that only then we can apply deep learning paradigms, and , furthermore, we can assign color code to every pixel.
We can utilize RGB color codes, where different values express various colors, each with a maximum of 255. Hence we convert all available pixel values from the range 0 to 255 to the range of 0 to 1, by simple division.
Lastly and optionally, we reduce the size of the training and testing data to speed up the experimentation.
Afterward, the model fitting begins. In the context of this research, the accuracy and the execution time are used to rate each algorithm.