\chapter{Discussion}
The programs and their results are sufficiently satisfactory.

The differences of each kernel are made clear thanks to the hyperparameter evaluation, albeit the outcome of the principal component analysis-supported datasets were surprising.
In comparison with the simple-reshaped datasets, the accuracy has risen while the time needed has worsened for most cases.
Given that the main idea of PCA is to explain variances between different data points and dimensions, it may not be optimal to use on the MNIST-Digit dataset which consists only of two-dimensional pixel images.
On the other hand, the neural networks have performed excellently, with accuracies rivaling that of the SVMs at far faster execution times. 

The results indicate that both, SVMs and neural networks, with and without PCA, are extraordinarily useful algorithms to recognize hand-written digits.
Overall the neural network performed better in terms of time (SKLearn-MLP: 0.0022s), while SVMs allowed for the highest possible accuracy (Poly-PCA: 0.9719).
It is noteworthy that there are perhaps far better results achievable, depending on the experimentation of parameters.
